\documentclass[titlepage]{article}
\usepackage{amsmath}
\usepackage{framed}
\usepackage[margin = 1in]{geometry}
\title{Linear Motion}
\author{Kevin Yang}
\date{}
\begin{document}
\maketitle
\section{Understanding Motion}
To begin understanding physics, we must first understand how the world is able to move around us. More specifically: straight line, aka linear, motion.

\subsection{1D Motion}
To make things simplier to begin with, let us first analyze how an object moves in only one direction. 
\\
Let's start by imagining how the real world works. Suppose we have an object at rest, aka not moving. 
In order to move it, we must push or pull the object. This push or pull on the object is what we call 
\begin{center}
	$\bigodot$ Force
\end{center}
Now that we exerted a force on the object, the object is now moving. If the ground was ice, then the object would not stop unless we exert a force to stop it. 
\\
The idea of an object remaining at rest or in motion unless a force is acted upon it is \textbf{Newton's First Law}.
\begin{center}
	\begin{framed}
		IFF $F_{net} = 0$ then $a = 0$
		\\
		$\bigodot$ IFF means "if and only if"
	\end{framed}	
\end{center}
\end{document}